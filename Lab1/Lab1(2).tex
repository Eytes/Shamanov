\documentclass[11pt]{article}
\usepackage{amsmath,amssymb,amsthm}
\usepackage{algorithm}
\usepackage[noend]{algpseudocode} 
\newcommand{\anonsection}[1]{\section*{#1}\addcontentsline{toc}{section}{#1}}
\usepackage{fancyhdr}
\setcounter{page}{148}

%---enable russian----

\usepackage[utf8]{inputenc}
\usepackage[russian]{babel}

% PROBABILITY SYMBOLS
\newcommand*\PROB\Pr 
\DeclareMathOperator*{\EXPECT}{\mathbb{E}}


% Sets, Rngs, ets 
\newcommand{\N}{{{\mathbb N}}}
\newcommand{\Z}{{{\mathbb Z}}}
\newcommand{\R}{{{\mathbb R}}}
\newcommand{\Zp}{\ints_p} % Integers modulo p
\newcommand{\Zq}{\ints_q} % Integers modulo q
\newcommand{\Zn}{\ints_N} % Integers modulo N

% Landau 
\newcommand{\bigO}{\mathcal{O}}
\newcommand*{\OLandau}{\bigO}
\newcommand*{\WLandau}{\Omega}
\newcommand*{\xOLandau}{\widetilde{\OLandau}}
\newcommand*{\xWLandau}{\widetilde{\WLandau}}
\newcommand*{\TLandau}{\Theta}
\newcommand*{\xTLandau}{\widetilde{\TLandau}}
\newcommand{\smallo}{o} %technically, an omicron
\newcommand{\softO}{\widetilde{\bigO}}
\newcommand{\wLandau}{\omega}
\newcommand{\negl}{\mathrm{negl}} 

% Misc
\newcommand{\eps}{\varepsilon}
\newcommand{\inprod}[1]{\left\langle #1 \right\rangle}

 
\newcommand{\handout}[5]{
  \noindent
  \begin{center}
  \framebox{
    \vbox{
      \hbox to 5.78in { {\bf Научно-исследовательская практика} \hfill #2 }
      \vspace{4mm}
      \hbox to 5.78in { {\Large \hfill #5  \hfill} }
      \vspace{2mm}
      \hbox to 5.78in { {\em #3 \hfill #4} }
    }
  }
  \end{center}
  \vspace*{4mm}
}

\newcommand{\lecture}[4]{\handout{#1}{#2}{#3}{Scribe: #4}{Эйлерово обобщение теоремы Ферма #1}}

\newtheorem{theorem}{Теорема}
\newtheorem{lemma}{Лемма}
\newtheorem{definition}{Определение}
\newtheorem{corollary}{Следствие}
\newtheorem{fact}{Факт}

% 1-inch margins
\topmargin 0pt
\advance \topmargin by -\headheight
\advance \topmargin by -\headsep
\textheight 8.9in
\oddsidemargin 0pt
\evensidemargin \oddsidemargin
\marginparwidth 0.5in
\textwidth 6.5in

\parindent 0in
\parskip 1.5ex

\begin{document}

\lecture{}{Лето 2020}{}{Юрий Шаманов}

	\pagestyle{fancy}
\rhead{CHAP. 7}
\chead{Эйлерово обобщение теоремы Ферма}
\lhead{\thepage}


$8$. Учитывая, что \textit{n} $\geq$ 1, набор \textit{$\phi(n)$} целых чисел, которые взаимно протые и которые являются несогласованными по модулю \textit{n}, \textit{называются уменьшенным набором остатков по модулю n} (то есть по уменьшенным набором остатков явдяются те члены полного набора остатков по модулю \textit{n}, которые относительно просты по модулю \textit{n}).

Убедитесть, что

\begin{enumerate}
	\begin{enumerate}
		
		\item целые чилса $-31, -16, -8, 13, 25, 80$ формируют уменьшенныйнабор остатков по модулю $9$;
		
		\item  целые чилса $3, 3^{2}, 3^{3}, 3^{4}, 3^{5}, 3^{6}$ формируют уменьшенный набор остатков по модулю $14$;

		\item  целые чилса $2, 2^{2}, 2^{3}, \ldots, 2^{18}$  формируют уменьшенный набор остатков по модулю $27$;
	\end{enumerate}
\end{enumerate}

$9$. Если \textit{p} –  простое число, покажите, что целые числа

\[
-\frac{\textit{p}-1}{2} , \ldots , -2, -1, 1, 2, \ldots , \frac{\textit{p}-1}{2}
\]

формируют уменьшенный набор остатков по модулю \textit{p};

\anonsection{
	\begin{flushleft}
		7.4 Некоторые свойства \\ функции Эйлера	
	\end{flushleft}
} 

Следующая теорема указываепт на любопытную особенность функции Эйлера, а именно, что сумма значений \textit{$\phi(d)$}, где \textit{d} — положителоьный делитель \textit{n}, равсна самой \textit{n}. Впервые это заметил Гаус.

%Теорема $7-6$ (Гаусc) \textit{для положительного целового числа  n} $\geq$ 1,
Теорема $7-6$  \begin{theorem}
	(Гаусc) для положительного целового числа  n $\geq$ 1,
	

\[
n = \sum_{d|n}\phi(d),
\]

\textit{сумма распространяется на все положительные делители n}.
\end{theorem}

\textit{Доказательство}: целы числа от $1$ до n можно разделить на классы следующим образом: если d – положительный делитель n, то мы помещаем целое число m в класс Sd при условии, что НОД(m,n) = d.

\[
S_{d} = \left\{m | \gcd(m, n) = d; 1 \leq m \leq n\right\}.	
\]


Теперь НОД(\textit{m, n}) = \textit{d} тогда и только тогда, если  НОД(\textit{m/d, n/d}) = $1$. Таким образом, количесво целых чисел в классе равно количесву положительных чисел, не превышающих  n/d и взаимно просты с n/d; другими словами, равно $\phi(n)$. Так как каждое из n положительных чисел в множестве $\left\{1, 2, … , n\right\}$ лежит ровно в одном классе S$_d$, мы получаем формулу

\[
n = \sum_{d|n}\phi(n/d)
\]


Но так \textit{d} проходит все положительные делители \textit{n} как и \textit{n/d}; следовательно, 

\[
\sum_{d|n}\phi(\textit{n/d}) = \sum_{d|n}\phi(\textit{d})
\]

далее следует теорема.

\begin{flushleft}
	\textbf{Пример $7-3$}
\end{flushleft}


Простой числовой пример того, что мы только что сказали, приведен при \textit{n} = $10$. Здесь классы такие:

\begin{center}

	 S$_1$  = $\left\{1, 3, 7, 9\right\}$,\\
	
	 S$_2$ = $\left\{2, 4, 6, 8\right\}$,\\
	
	 S$_5$ = $\left\{5\right\}$,\\
	
	 S$_{10}$ = $\left\{10\right\}$.
	
\end{center}				


Они содержат $\phi(10) = 4, \phi(5) = 4, \phi(2) = 1, \phi(1) = 1$ целых числех. Следовательно, 

\[
	\sum_{d|10}\phi(\textit{d}) = \phi(10) + \phi(5) + \phi(2) + \phi(1) = 4 + 4 + 1 + 1 = 10
\]

Поучительно привести второе доказательство теоремы $7-6$, на этот раз в зависимости от того, что  $\phi$ — мультипликативна. Подробнее: если \textit{n} = $1$,  тогда все ясно 

\[
\sum_{d|n}\phi(n/d) = \sum_{d|1}\phi(d) = \phi(1) = 1 = \textit{n}
\]

Предположим, что $n \geq 1$, рассмотрим теоретически-числовую фунцкцию

\[
F(n) = \sum_{d|n}\phi(d)
\]

Поскольку $\phi$ — мультипликативная функция , тогда по теореме $6-4$ F – мультипликативна. Следовательно, если $n = p_1^{k_{1}}p_2^{k_{2}} \ldots p_r^{k_{r}}$  является главной факторизации n, тогда 

\[
F(n) = F(p_1^{k_{1}})F(p_2^{k_{2}})\ldots F(p_r^{k_{r}})
\]

Для каждого значения i,

$
F(p_i^{k_{i}}) = \sum_{d|p_i^{k_{i}}}\phi(d)
$
	


\begin{align*}
	& = \phi(1) + \phi(p_i) + \phi(p_i^2) + \phi(p_i^3)+ \ldots + \phi(p_i^{k_{i}}) \\
	& = 1 + (p_i – 1) + (p_i^2  - p_i) + (p_i^3  - p_i^2 ) + \ldots + (p_i^{k_{i}}  - p_i^{k_{i} - 1}) \\
	& = p_i^{k_{i}},
\end{align*}

термины в приведенном выше выражении исключают друг друга, за исключением термина  $p_i^{k_{i}}$. Зная это, мы в конечном итоге получаем

\[
	F(n) = p_1^{k_{1}}p_2^{k_{2}}\ldots p_r^{k_{r}} = \textit{n}
\]

и так же 

\[
	n = \sum_{d|n}\phi(d)
\]

Отметим так же, что есть еще одино интересное тождество, которое включает в себя вункцию Эйлера.

\begin{theorem}
	Для  n > $1$, сумма положительных целых чисел меньше n и взаимно простых с n, это
	 $\frac{1}{2}n\phi(n)$,

\[
	\frac{1}{2}n\phi(n) = \sum\limits_{\begin{split} \gcd(k,& n) = 1\\ 1 \leq k& < n\end{split}}k.
\]
\end{theorem}

\textit{Доказательство:} пусть \textit{ а$_1$, а$_2$, … , а$_{\phi(n)}$}  положительные целые числа меньше \textit{n} и взаимно простые с n. Теперь НОД(a, n) = $1$ тогда и только тогда НОД(\textit{n - a, n}) = $1$, тогда мы имеем 

\begin{align*}
a_1  + a_2 +  \ldots  + a_{\phi(n)} & = (n - a_1) + (n - a_2) + \ldots + (n - a_{\phi(n)}) \\
& = \phi(n)n - (a_1  + a_2 + \ldots  + a_{\phi(n)}).
\end{align*}

Следовательно, 

\[
2( a_1 + a_2 + \ldots + a_{\phi(n)} )= \phi(n) n
\]


подводя к заявленному выводу.

\begin{flushleft}
	\textbf{Пример $7-4$}
\end{flushleft}



Рассмотрим случай \textit{n} = $30$. $\phi(30)$ = $8$ целых чисел, которые меньше $30$ и взаимно простые с ним

\begin{center}
	$1, 7, 11, 13, 17, 19, 23, 29$.
\end{center}





\end{document}