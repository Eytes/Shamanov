\documentclass[12pt]{article}
\usepackage{cmap}
\usepackage[T2A]{fontenc}
\usepackage[utf8]{inputenc}
\usepackage[english,russian]{babel}
\usepackage[left=1cm,right=1cm,
top=2cm,bottom=2cm,bindingoffset=0cm]{geometry}
\newcommand{\anonsection}[1]{\section*{#1}\addcontentsline{toc}{section}{#1}}

\usepackage{fancyhdr}
\setcounter{page}{148}

\begin{document}
	
	\pagestyle{fancy}
	\rhead{CHAP. 7}
	\chead{Эйлерово обобщение теоремы Ферма}
	\lhead{\thepage}
	
	
	$8$. Учитывая, что \textit{n} $\geq$ 1, набор \textit{$\phi(n)$} целых чисел, которые взаимно протые и которые являются несогласованными по модулю \textit{n}, \textit{называются уменьшенным набором остатков по модулю n} (то есть по уменьшенным набором остатков явдяются те члены полного набора остатков по модулю \textit{n}, которые относительно просты по модулю \textit{n}).
	
	Убедитесть, что
	
	(a)  целые чилса $-31, -16, -8, 13, 25, 80$ формируют уменьшенныйнабор остатков по модулю $9$;
	
	(б)  целые чилса $3, 3^{2}, 3^{3}, 3^{4}, 3^{5}, 3^{6}$ формируют уменьшенный набор остатков по модулю $14$;
	
	(в)  целые чилса $2, 2^{2}, 2^{3}, …, 2^{18}$  формируют уменьшенный набор остатков по модулю $27$;
	
	$9$. Если \textit{p} –  простое число, покажите, что целы числа
	
	\begin{center}	$-\frac{\textit{p}-1}{2}$ , … , $-2, -1, 1, 2$, … , $\frac{\textit{p}-1}{2}$
	\end{center}
	
	формируют уменьшенный набор остатков по модулю \textit{p};
	
	\anonsection{
		\begin{flushleft}
			7.4 Некоторые свойства \\ функции Эйлера	
		\end{flushleft}
	} 
	
	Следующая теорема указываепт на любопытную особенность функции Эйлера, а именно, что сумма значений \textit{$\phi(d)$}, где \textit{d} — положителоьный делитель \textit{n}, равсна самой \textit{n}. Впервые это заметил Гаус.
	
	Теорема $7-6$ (Гаусc) \textit{для положительного целового числа  n} $\geq$ 1,
	
	\begin{center}
		$n = \sum_{d|n}\phi(d),$	
	\end{center}
	
	
	\textit{сумма распространяется на все положительные делители n}.
	
	\textit{Доказательство}: целы числа от $1$ до n можно разделить на классы следующим образом: если d – положительный делитель n, то мы помещаем целое число m в класс Sd при условии, что НОД(m,n) = d.
	
	\begin{center}
		$S_{d}$ = $\left\{m | НОД(m, n) = d; 1 \leq m \leq n\right\}$.	
	\end{center}
	
	
	Теперь НОД(\textit{m, n}) = \textit{d} тогда и только тогда, если  НОД(\textit{m/d, n/d}) = $1$. Таким образом, количесво целых чисел в классе равно количесву положительных чисел, не превышающих  n/d и взаимно просты с n/d; другими словами, равно $\phi(n)$. Так как каждое из n положительных чисел в множестве $\left\{1, 2, … , n\right\}$ лежит ровно в одном классе S$_d$, мы получаем формулу
	
	\begin{center}
		$n = \sum_{d|n}\phi(n/d)$.	
	\end{center}
	
	
	Но так \textit{d} проходит все положительные делители \textit{n} как и \textit{n/d}; следовательно, 
	
	\begin{center}
		$\sum_{d|n}\phi(\textit{n/d}) = \sum_{d|n}\phi(\textit{d})$	
	\end{center}
	
	
	далее следует теорема.
	
	\begin{flushleft}
		\textbf{Пример $7-3$}
	\end{flushleft}
	
	
	Простой числовой пример того, что мы только что сказали, приведен при \textit{n} = $10$. Здесь классы такие:
	
	\begin{center}
		S$_1$ = $\left\{1, 3, 7, 9\right\}$,
		
		S$_2$ = $\left\{2, 4, 6, 8\right\}$,
		
		S$_5$ = $\left\{5\right\}$,
		
		S$_{10}$ = $\left\{10\right\}$.	
	\end{center}				
	
	
	Они содержат $\phi(10) = 4, \phi(5) = 4, \phi(2) = 1, \phi(1) = 1$ целых числех. Следовательно, 
	
	\begin{center}
		$\sum_{d|10}\phi(\textit{d}) = \phi(10) + \phi(5) + \phi(2) + \phi(1) = 4 + 4 + 1 + 1 = 10$.	
	\end{center}
	
	
	Поучительно привести второе доказательство теоремы $7-6$, на этот раз в зависимости от того, что  $\phi$ — мультипликативна. Подробнее: если \textit{n} = $1$,  тогда все ясно 
	
	\begin{center}
		$\sum_{d|n}\phi(n/d) = \sum_{d|1}\phi(d) = \phi(1) = 1 = \textit{n}$	
	\end{center}
	
	
	Предположим, что $n \geq 1$, рассмотрим теоретически-числовую фунцкцию
	
	\begin{center}
		$F(n) = \sum_{d|n}\phi(d)$.	
	\end{center}
	
	
	Поскольку $\phi$ — мультипликативная функция , тогда по теореме $6-4$ F – мультипликативна. Следовательно, если $n = p_1^{k_{1}}p_2^{k_{2}} … p_r^{k_{r}}$  является главной факторизации n, тогда 
	
	\begin{center}
		$F(n) = F(p_1^{k_{1}})F(p_2^{k_{2}})...F(p_r^{k_{r}})$ .	
	\end{center}
	
	
	Для каждого значения i,
	
	$F(p_i^{k_{i}}) =  \sum_{d|p_i^{k_{i}}}\phi(d)$	
	
	\begin{center}
		= $\phi(1)$ + $\phi(p_i)$ + $\phi(p_i^2)$ + $\phi(p_i^3)+ … + \phi(p_i^{k_{i}})$ \\
		= $1 + (p_i – 1) + (p_i^2  - p_i) + (p_i^3  - p_i^2 ) + … + (p_i^{k_{i}}  - p_i^{k_{i} - 1})$ \\
		= $p_i^{k_{i}}$,	
	\end{center}
	
	
	термины в приведенном выше выражении исключают друг друга, за исключением термина  $p_i^{k_{i}}$. Зная это, мы в конечном итоге получаем
	
	\begin{center}
		$F(n) = p_1^{k_{1}}p_2^{k_{2}}...p_r^{k_{r}} = \textit{n}$	
	\end{center}
	
	
	и так же 
	
	\begin{center}
		$n = \sum_{d|n}\phi(d)$.	
	\end{center}
	
	
	Отметим так же, что есть еще одино интересное тождество, которое включает в себя вункцию Эйлера.
	
	Теорема $7-7$. \textit{Для  n > $1$, сумма положительных целых чисел меньше n и взаимно простых с n, это} $\frac{1}{2}n\phi(n)$,
	
	\begin{center}
		$\frac{1}{2}n\phi(n) = 
		\sum_{НОД(k, n) = 1, 1 \leq k < n}k$	
	\end{center}
	
	
	\textit{Доказательство:} пусть \textit{ а$_1$, а$_2$, … , а$_{\phi(n)}$}  положительные целые числа меньше \textit{n} и взаимно простые с n. Теперь НОД(a, n) = $1$ тогда и только тогда НОД(\textit{n - a, n}) = $1$, тогда мы имеем 
	
	\begin{center}
		а$_1$  + а$_2$ +  …  + а$_{\phi(n)}$  = (n – а$_1$) + (n – а$_2$) + … + (\textit{n} – а$_{\phi(n)}$) \\
		= $\phi(n)$n - (а$_1$  + а$_2$ +  …  + а$_{\phi(n)}$).	
	\end{center}
	
	
	Следовательно, 
	
	\begin{center}
		$2$(а$_1$  + а$_2$ +  …  + а$_{\phi(n)}$ ) =  $\phi(n)$\textit{n},
	\end{center}
	
	
	подводя к заявленному выводу.
	
	\begin{flushleft}
		\textbf{Пример $7-4$}
	\end{flushleft}
	
	
	
	Рассмотрим случай \textit{n} = $30$. $\phi(30)$ = $8$ целых чисел, которые меньше $30$ и взаимно простые с ним
	
	\begin{center}
		$1, 7, 11, 13, 17, 19, 23, 29$.
	\end{center}
	
\end{document}
